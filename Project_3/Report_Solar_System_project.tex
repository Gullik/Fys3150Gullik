
\documentclass[11pt,a4wide]{article}
\usepackage{verbatim}
\usepackage{listings}
\usepackage{graphicx}
\usepackage{a4wide}
\usepackage{color}
\usepackage{amsmath}
\usepackage{amssymb}
\usepackage[dvips]{epsfig}
\usepackage[T1]{fontenc}
\usepackage{cite} % [2,3,4] --> [2--4]
\usepackage{shadow}
\usepackage{hyperref}
\usepackage{physics}

\setcounter{tocdepth}{2}

\lstset{language=c++}
\lstset{alsolanguage=[90]Fortran}
\lstset{basicstyle=\small}
\lstset{backgroundcolor=\color{white}}
\lstset{frame=single}
\lstset{stringstyle=\ttfamily}
\lstset{keywordstyle=\color{red}\bfseries}
\lstset{commentstyle=\itshape\color{blue}}
\lstset{showspaces=false}
\lstset{showstringspaces=false}
\lstset{showtabs=false}
\lstset{breaklines}

\title{ Project 1 in FYS-3150 }
\author{Gullik Vetvik Killie }

\begin{document}

\maketitle

\tableofcontents

\newpage

\section{Task a: Prepare the problem, implement Runge-Kutta 4}
	\subsection{Preparing simple ODE}
		For the first part of the project we will set up a simple system system of just the Earth orbiting the Sun, with the Sun 
		acting as a constant anchor unnaffected by the earth since its mass is several orders of magnitude larger than the Earth, \(M_S >> M_E\).
		This simple classical system is governed by Newtons laws. We will use index notation with \(\pdv[2]{x}{t} = \partial^2 x\), 
		with alle derivatives being with respect to time unless spesified otherwise. If an index repeats summation over \(i\) is implied, 
		so for \(i = 1,2,3\), it will mean:
		\( x_ix_i = x_1x_1 + x_2x_2+x_3x_3\)
		
		From Newtons second law we get:
		
		\begin{align}
			M_E\partial^2{x_i} &= F_i
			\intertext{The only force acting on the earth is the gravitational force}
			\partial^2{x_i} &= \frac{x_i}{(x_jx_j)^{1/2}} \frac{(GM_SM_E)/(x_jx_j)}{M_E}
			\\
			\partial^2{x_i} &= x_i\frac{GM_S}{(x_jx_j)^{3/2}}
			\label{eq:one_object}
		\end{align}
		
		Now we will divide it into a coupled set of two first order ODE's
		
		\begin{subequations}
			\begin{align}
			y_i^{(1)} & = x_i
			\\
			y_i^{(2)} & = \partial y_i^{(1)}
			\end{align}
			\label{eq:subdivide}
		\end{subequations}
		
		We then insert our new equationset, \eqref{eq:subdivide}, into equation \eqref{eq:one_object} and obtain
		
		\begin{subequations}
			\begin{align}
			\partial y_i^{(1)} &= y_i^{(2)}
			\\
			\partial y_i^{(2)} &= y_i^{(1)} \frac{GM_S}{(y_j^{(1)}y_j^{(1)})^{3/2}}
			\end{align}
		\end{subequations}
		
		Now we have a set of 2 coupled first order ODE's, with 2 unknowns, and we will now discretize them.
		
	\subsection{Runge-Kutta 4 method}
		The Runge-Kutta 4 method is based on using 4 steps to calculate the next iteration of the function we approximate.
		Given a that the derivative of a function is given by a general function \(f(t,y\)
		
		\begin{align}
			y(t) &= \int{f(t,y)}dt
			\intertext{Given that \(y_k\) is known the next point, \(y_{k+1}\) on the function is given by}
			y_{k+1} &= y_{k} + \int^{t+1}_{t}f(t,y)dt
				\label{eq:Runge_Kutta}
			\intertext{Let \(h\) be the stepsize and then a Taylor expand around half the step, \(t+1/2\), results in }
			f(t,y) & = f(t_{i+1/2},y_{i+1/2}) + (t-t_{i+1/2})\dv{}{t} (f(t_{i+1/2},y_{i+1/2}))  + O(h^3)
			\intertext{We then integrate this from \(t_i \to t_{i+1}\) using a midpoint rule and get}
			\int^{t+1}_{t}f(t,y)dt &\approx hf(t_{i+1/2},y_{i+1/2})
			\intertext{Inserting this into equation \eqref{eq:Runge_Kutta}}
			y_{k+1} &= y_{k} + hf(t_{i+1/2},y_{i+1/2}) + O(h^3)
			\intertext{Then a forward Euler method is used to approximate \(y_{i+1/2} \approx y_i + \frac{h}{2} \dv{y}{t} 
				= y_i + \frac{h}{2}f(t_i,y_i)\)}
			s
		\end{align}
		
		
\end{document}
